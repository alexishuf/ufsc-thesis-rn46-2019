\documentclass[embeddedlogo]{ufsc-thesis-rn46-2019}

\usepackage[utf8]{inputenc} % UTF-8
\usepackage{lipsum} % Gerador de texto
\usepackage{pdfpages} % Inclui PDF externo (ficha catalográfica)

% Usado para mostrar código
\usepackage[cache=false]{minted}
\newmintinline[mt]{latex}{fontsize=\normalsize}
\setminted{fontsize=\tiny,linenos,xleftmargin=2em}
\setmintedinline{breaklines,breakbytokenanywhere}

%%%%%%%%%%%%%%%%%%%%%%%%%%%%%%%%%%%%%%%%%%%%%%%%%%%%%%%%%%%%%%%%%%%%
%%% Configurações da classe (dados do trabalho)                  %%%
%%%%%%%%%%%%%%%%%%%%%%%%%%%%%%%%%%%%%%%%%%%%%%%%%%%%%%%%%%%%%%%%%%%%

% Preâmbulo
\titulo{Template \LaTeX{} seguindo a RN 46/2019/CPG da UFSC}
\autor{Omar Ravenhurst}
% Importante! Para documentos em inglês, não use today, digite a data em
% pt_BR, como deve aparecer na folha de certificação.
\data{1 de Agosto de 2019}
\instituicao{Universidade Federal de Santa Catarina}
\programa{Programa de Pós-Graduação em Ciência da Computação}
\tese % ou \dissertacao
\local{Florianópolis} % Apenas cidade! Sem estado
\titulode{Doutor em Ciência da Computação}
\orientador{Prof. Dr. Ben Trovato}
\coorientador{Prof. Dr. Lars Thørväld}
\centro{Centro Tecnológico}

% Membros da banca e coordenador
% As regras da BU agora exigem que Dr. apareça depois do nome
\membrobanca{Prof. Valerie Béranger, Dr.}{Universidade Federal de Santa Catarina}
\membrobanca{Prof. Aparna Patel, Dr.}{Universidade Federal de Santa Catarina}
\membrobanca{Prof. Huifen Chan, Dr.}{Universidade Federal de Santa Catarina}
% Atenção! o template da BU e o documento que apresenta as regras continua
% usando Dr antes do nome para Orientador e Coordenador!
\coordenador{Prof. Dr. Charles Palmer}


\begin{document}

%%%%%%%%%%%%%%%%%%%%%%%%%%%%%%%%%%%%%%%%%%%%%%%%%%%%%%%%%%%%%%%%%%%%
%%% Principais elementos pré-textuais                            %%%
%%%%%%%%%%%%%%%%%%%%%%%%%%%%%%%%%%%%%%%%%%%%%%%%%%%%%%%%%%%%%%%%%%%%

% Inicia parte pré-textual do documento capa, folha de rosto, folha de
% aprovação, aprovação, resumo, lista de tabelas, lista de figuras, etc.
\pretextual%
\imprimircapa%
\imprimirfolhaderosto*
\protect\incluirfichacatalografica{ficha.pdf}
\imprimirfolhadecertificacao
\clearpage \listoffigures*
\clearpage \tableofcontents*%

%%%%%%%%%%%%%%%%%%%%%%%%%%%%%%%%%%%%%%%%%%%%%%%%%%%%%%%%%%%%%%%%%%%%
%%% Corpo do texto                                               %%%
%%%%%%%%%%%%%%%%%%%%%%%%%%%%%%%%%%%%%%%%%%%%%%%%%%%%%%%%%%%%%%%%%%%%
\textual%

\chapter{Introdução}

Bem-vindo ao guia de usuário da classe \texttt{ufsc-thesis-rn46-2019}. Essa
classe é um conjunto de customizações aplicadas à classe
\href{https://ctan.org/pkg/abntex2}{\abnTeX} e ao pacote \texttt{abntex2cite}.
O objetivo da classe \texttt{ufsc-thesis-rn46-2019} é simplório: adequar o
\abnTeX{} às \href{http://portal.bu.ufsc.br/normalizacao/}{normas emitidas pela
Biblioteca Universitária da UFSC} em sequência à
\href{https://repositorio.ufsc.br/handle/123456789/197121}{Resolução Normativa
nº 46/2019/CPG}.


\section{Autores, suporte e atualizações}

Essa classe foi escrita incialmente por dois alunos do
\href{http://ppgcc.posgrad.ufsc.br/}{PPGCC da UFSC}: Alexis Huf e Gustavo
Zambonin. Há o risco de esse arquivo não ser atualizado a cada \textit{pull
request}, então confira a lista de mártires no GitHub. Essa classe é mantida no
repositório
\href{https://github.com/alexishuf/ufsc-thesis-rn46-2019/}{alexishuf/ufsc-thesis-rn46-2019}.
Atualizações podem ser encontradas nesse repositório. \textit{Issues} e PRs
são bem vindos.

\subsection{Registro de mudanças}

Lista de versões (pelo menos das versões que receberam um número):
\begin{description}
\item[v0.1-alpha] (2019-08-01) Primeira versão para ajudar alunos com pouco
    prazo de entrega.
\end{description}

\section{Guia rápido}
\label{sec:quick}

A classe \texttt{ufsc-thesis-rn46-2019} deveria ser encarada como um
\emph{drop-in} para o \abnTeX. Você pode começar a escrever uma tese do zero
usando apenas esta classe, mas talvez você queira usar algum template indicado
por algum colega. Na maioria dos casos, bastará usar
\mt|\documentclass{ufsc-thesis-rn46-2019}|, ajustando o caminho se necessário.
Há duas principais formas de incluir essa classe no seu projeto.

\begin{enumerate}
  \item Copie o arquivo \texttt{ufsc-thesis-rn46-2019.cls}. Sim, isso funciona
      pois o logo da UFSC está embutido em base64 dentro do cls. Para que isso
        funcione, você deve compilar com \texttt{-shell-escape} em um ambiente
        \textit{UNIX-like}, como no arquivo \texttt{Makefile} de exemplo.
  \item Copie a pasta do repositório do GitHub para dentro do seu projeto.
      Nesse caso você deverá adicionar o caminho dessa pasta ao fazer o
        \mt|\documentclass|. Há 3 formas de incluir essa pasta:
  \begin{enumerate}
    \item Use a opção ``\emph{Download ZIP}'' do GitHub se você não sabe o que
        é Git.
    \item Via \texttt{git submodule}, se você já está usando Git.
    \item Via \texttt{git clone} se não estiver usando Git no seu projeto
        (\emph{shame on you}) ou se estiver usando no Overleaf, onde submódulos
          não funcionam.
    \end{enumerate}
\end{enumerate}

Esse guia rápido assume que você está começando do zero, sem um template
anterior, com intuito de ser didático. Se o seu trabalho for escrito em língua
inglesa, você deverá passar uma opção \mt|english| para a classe. Caso
contrário não precisa passar nenhuma opção. Logo após chamar a classe, você
deverá fornecer dados para a classe (e para o \abnTeX). Veja um preâmbulo
completo de exemplo na \autoref{fig:preambulo}. Os próximos parágrafos. Segue
um rápido overview sobre o significado de cada um dos comandos usados (a maior
parte deles continua funcionando como no \abnTeX).

\begin{itemize}
  \item \mt|\titulode|, \mt|\autor|, \mt|\instituicao|, \mt|\orientador| e
      \mt|\coorientador|: esses comandos continuam com o mesmo significado e
        uso que possuem no \abnTeX. Para facilitar o uso, também foram
        adicionados \mt|\orientadora| e \mt|\coorientadora|.
  \item \mt|\data|: as regras da UFSC exigem que apenas o ano esteja presente
      na capa e folha de rosto. No entanto, a nova folha de certificação, que é
        gerada por essa classe precisa da data completa. Forneça a data
        completa \textbf{em português} (mesmo para documentos em inglês). A
        classe irá extrair o ano.
  \item \mt|\programa| e \mt|\centro|: nome do Programa de Pós-Graduação, por
      extenso, e nome do centro, \emph{e.g.} Centro Tecnológico.
  \item \mt|\tese| e \mt|\dissertacao|: o texto a ser colocado abaixo do título
      na folha de rosto tem regras bem definidas. Você deve usar um desses dois
        comandos para indicar o tipo de trabalho (e consequentemente o nível).
        Evidentemente, são mutuamente exclusivos.
  \item \mt|\preambulo|: Esse comando fornecido pelo \abnTeX{} é desnecessário
      nessa classe. Entretanto, você pode usar ele para sobrescrever o texto
        gerado automaticamente a partir do uso de \mt|\tese| e \mt|\titulode|.
\end{itemize}

A ``folha de certificação'', que substituiu a antiga folha de aprovação, agora
é gerada em \LaTeX. O nome do orientador já foi fornecido anteriormente, resta
apenas indicar os membros avaliadores da banca e o coordenador do programa:
\begin{itemize}
  \item \mt|\membrobanca{<nome>}{<instituição>}|: adiciona um membro da banca.
      \textbf{Atenção}: para membros avaliadores, o ``Dr.'' deve ser inserido
        após o nome e o ``Prof(a).'' deve preceder o nome.
  \item \mt|\coordenador| e \mt|\coordenadora|: Configura o nome do(a)
      coordenador(a) do Programa de Pós-Graduação.
\end{itemize}

\begin{figure}[tb]
  \centering
  \caption{Preâmbulo de uma tese (ou dissertação) típica usando esta classe.}
  \label{fig:preambulo}

  \begin{minted}{latex}
\documentclass[english]{ufsc-thesis-rn46-2019/ufsc-thesis-rn46-2019}

% \usepackage's a gosto

\titulo{Template \LaTeX~ seguindo a RN 46/2019/CPG da UFSC}
\autor{Fulano da Silva}
\data{1 de Agosto de 2019}
\instituicao{Universidade Federal de Santa Catarina}
\programa{Programa de Pós-Graduação em Ciências da Computação}
\local{Florianópolis} % Apenas cidade! Sem estado
\tese % ou \dissertacao
\titulode{Doutor em Ciência da Computação}
\orientador{Prof. Dr. Ben Trovato}
\coorientador{Prof. Dr. Lars Thørväld}
\centro{Centro Tecnológico}

\membrobanca{Prof. Valerie Béranger, Dr.}{Universidade Federal de Santa Catarina}
\membrobanca{Prof. Aparna Patel, Dr.}{Universidade Federal de Santa Catarina}
\membrobanca{Prof. Huifen Chan, Dr.}{Universidade Federal de Santa Catarina}
\coordenador{Prof. Dr. Charles Palmer}
  \end{minted}
  \captionsource{o autor.}
\end{figure}

Os elementos pré-textuais permanecem em grande parte tendo seu comportamento
determinado pelo \abnTeX. Alguns elementos incluídos nesse guia foram alterados
para satisfazer normas da UFSC. Esses elementos pré-textuais também presentes
nesse guia são obtidos com o código mostrado na \autoref{fig:pre}. Capa
(\mt|\imprimircapa|), folha de rosto (\mt|\imprimirfolhaderosto*|) listas
(\emph{e.g.} \mt|\listoffigures*|) e o sumário \mt|\tableofcontents*| são
obtidos através dos mesmos comandos do \abnTeX, mas agora respeitam novas
regras impostas pela BU.

A ficha catalográfica deve ser gerada em sistema próprio da BU, que produz um
PDF. Esse PDF deve ser salvo no seu projeto e incluído no documento. A macro
\mt|\incluirfichacatalografica{<pdf>}| faz essa inclusão usando o pacote
\href{https://www.ctan.org/pkg/pdfpages}{\mt|pdfpages|}. O \mt|\protect| é
necessário para evitar um \emph{bug} (uma linha em branco substitui o
\mt|\protect|).  A folha de certificação será gerada pelo comando
\mt|\imprimirfolhadecertificacao| de acordo (e na ordem) com os comandos de
dados emitidos anteriormente.

\begin{figure}[tb]
  \centering
  \caption{Elementos pré-textuais.}
  \label{fig:pre}
  \begin{minted}{latex}
\pretextual%
\imprimircapa%|*
\protect\incluirfichacatalografica{ficha.pdf}
\cleardoublepage \imprimirfolhadecertificacao
\cleardouplepage \listoffigures*
\cleardoublepage \tableofcontents*
  \end{minted}
  \captionsource{o autor.}
\end{figure}

Antes de iniciar o primeiro capítulo é necessário chamar a macro \mt|\textual|,
como em todo documento \abnTeX. Após os elementos textuais, os pós-textuais
devem ser precedidos por um \mt|\postextual|. Para fazer uso da bibliografia,
basta fazer como mostra a \autoref{fig:pos}. A classe já inclui o
\texttt{abntex2cite} e o configura. A UFSC impôs poucas alterações no
tratamento de bibliografia.

\begin{figure}[tb]
  \centering
  \caption{Elementos pós-textuais.}
  \label{fig:pos}
  \begin{minted}{latex}
\postextual
\bibliography{example}
  \end{minted}
  \captionsource{o autor.}
\end{figure}

\chapter{Referência de opções e macros}
\label{ch:ref}

\section{Opções da classe}

A classe possui apenas uma opção que fornece recursos, e todas as demais servem
para desativar comportamentos padrão.

\paragraph*{\texttt{english}} Prepara o \abnTeX{} e a classe para que o corpo
do documento esteja em inglês. Alguns elementos em português, especialmente os
relacionados a BU e a UFSC permanecem em português.

\paragraph*{\texttt{embeddedlogo}} Se ativada, essa opção desempacota o logo da
UFSC diretamente de um base64 dentro do arquivo \texttt{.cls}. Isso permite que
o \texttt{.cls} seja um arquivo auto-contido. Pessoas aversas a \texttt{git
submodule} e a diretórios podem simplesmente levar o \texttt{.cls} de um
projeto para outro. Entretanto, há uma desvantagem, e por isso essa opção não é
padrão. É necessário compilar com a opção \texttt{-shell-escape} em uma
ambiente \emph{UNIX-like}.

\paragraph*{\texttt{brazil}} Não tem efeito pois o idioma por padrão é o
português do Brasil. Entretanto, note que para utilizar outros idiomas além de
inglês e português, é necessário repassar todos os idiomas explicitamente à
classe \abnTeX.

\paragraph*{\texttt{times}, \texttt{lmodern}, \texttt{arial}} Escolhe a fonte
do documento. O padrão é \texttt{times}. No caso das escolhas \texttt{times} e
\texttt{arial} são usados clones \emph{open-source} (sem limitações legais à
redistribuição) que podem ser encontradas em qualquer instalação \LaTeX. A
opção \texttt{lmodern} corresponde à fonte Latin Modern, a fonte padrão do
\LaTeX. Essa é uma fonte serifada similar à Times, mas não possui respaldo
oficial da BU, mesmo sendo mais agradável e estilosa.

\paragraph*{\texttt{nocapautoref}} Essa classe configura o comando
\mt{\autoref} para que nomes de capítulos e seções sejam capitalizados. Ao
fornecer essa opção essa alteração deixará de ser feita.

\paragraph*{\texttt{noabntexcite}} Essa classe inclui o pacote
\texttt{abntex2cite} e o configura automaticamente. Caso seja necessário fazer
essa inclusão manualmente para fazer sua própria configuração, use essa opção
para que a classe não carregue o pacote.

\paragraph*{\texttt{nohidelinks}} O
\href{https://ctan.org/pkg/hyperref}{\texttt{hyperref}} por padrão desenha
bordas coloridas ao redor de \emph{hyperlinks}. Por padrão a classe desativa
esse comportamento, passando a opção \texttt{hidelinks} ao \texttt{hyperref}.
Ao ativar essa opção o \texttt{hyperref} voltará ao seu próprio comportamento
padrão.

\paragraph*{\texttt{nocleanheader}} O \abnTeX{} gera cabeçalhos na página que
incluem o título da seção. Essa classe suprime esse estilo de página,
proveniente da classe base \texttt{memoir}, pois eles não existem no template
da BU. Ao passar essa opção o cabeçalho padrão do \abnTeX{} será restaurado.

\paragraph*{\texttt{*}} Quaisquer outras opções não listadas anteriormente são
repassadas à classe \abnTeX{} sem manipulação.

\section{Novas macros}

Essa seção apresenta macros novas, não presentes no \abnTeX{} ou
\texttt{abntex2cite}.

\subsection{Comandos de dados}

Esses comandos visam fornecer dados que são posteriormente usados pelos
comandos de saída discutidos na próxima seção. \textbf{Importante:} não esqueça
de preencher os comandos de dados do próprio \abnTeX, como mostrado na
\autoref{sec:quick}.

\paragraph*{\texttt{\textbackslash{}programa\{<nome>\}}} Configura o nome do
Programa de Pós-Graduação.

\paragraph*{\texttt{\textbackslash{}titulode\{<texto>\}}} Define o título alvo.
O título \textit{deve ser escrito incluindo a especialidade}. Por exemplo,
\emph{Mestre em Ciência da Computação}.

\paragraph*{\texttt{\textbackslash{}tese}} Comando sem argumentos. Marca o
documento como uma tese de doutorado. Isso irá alterar o preâmbulo e a folha de
certificação.

\paragraph*{\texttt{\textbackslash{}dissertacao}} Comando sem argumentos. Marca
o documento como uma dissertação de mestrado. Esse é o padrão quando nem
\mt|\dissertacao| nem \mt|\tese| são usados.

\paragraph*{\texttt{\textbackslash{}nivel\{<nivel>\}}} Deve ser o nível do
título almejado, em minúsculas. O valor padrão é inferido a partir do uso (ou
não uso) de \mt|\tese| e \mt|\dissertacao|.

\paragraph*{\texttt{\textbackslash{}centro\{<texto>\}}} Centro da UFSC onde
está sediado o Programa de Pós-Graduação.

\paragraph*{\texttt{\textbackslash{}data\{<data por extenso>\}}} Define a data
a ser usada na folha de certificação. O ano dessa data será usado na capa e na
folha de rosto. Se esse comando for omitido, a data atual do sistema será
utilizada. Não utilize \mt|\today| como argumento em documentos em inglês. Isso
fará com que a data da folha de aprovação (que é em português) apareça em
inglês! Exemplo: \mt|\data{1 de Agosto de 2019}|.

\paragraph*{\texttt{\textbackslash{}ano\{<ano>\}} \\
            \texttt{\textbackslash{}mes\{<nome do mes>\}} \\
            \texttt{\textbackslash{}dia\{<dia>\}}}
Define cada componente da data individualmente.

\paragraph*{\texttt{\textbackslash{}membrobanca\{<nome com
títulos>\}\{<universidade por extenso>\}}}
Nome do professor avaliador da banca, incluíndo Prof. (ou Profa.) antes do
nome e o título \textbf{após} o nome. Exemplo: \emph{Prof. Mordecai
Malignatus, Dr.}. O orientador não deve ser incluído nessa seção. Não está
claro nas instruções da UFSC se o coorientador também não deveria. Note que
esse é o único lugar onde o título deve vir após o nome. Para orientador e
coordenador a BU coloca o título antes.

\paragraph*{\texttt{\textbackslash{}coordenadora\{<nome>\}}} Nome do
coordenador do programa. No caso de coordenadora, o comando \mt|\coordenadora|
pode ser usado. Nesse caso a BU usa títulos antes do nome.  Exemplo:
\mt|\coordenador{Prof. Dr. Mordecai Malignatus}|.

\subsection{Comandos de saída}

Esses comandos produzem resultados visíveis. Devem ser usados dentro do
ambiente \texttt{document}.

\paragraph*{\texttt{\textbackslash{}captionsource\{<texto>\}}} Insere a
descrição da fonte \texttt{Fonte: <texto>} que deve estar presente abaixo de
figuras, tabelas e quaisquer outros elementos deste tipo. Esse comando não
altera o alinhamento (veja a \autoref{sec:quick}).

\paragraph*{\texttt{\textbackslash{}incluirfichacatalografica\{<arquivo>\}}}
Inclui a ficha catalográfica proveniente de \texttt{arquivo} que foi gerado em
\href{http://ficha.bu.ufsc.br/}{ficha.bu.ufsc.br}.

\paragraph*{\texttt{\textbackslash{}imprimirfolhadecertificacao}} Imprime a
folha de certificação, que substitui a antiga folha de aprovação. Nessa folha
os membros da banca serão listados na ordem em que foram adicionados com o
comando \mt|\membrobanca{<nome>}{<universidade>}|.

\chapter{Mais exemplos de formatação}
\label{ch:ex}

Essa frase é verdadeira pois tem um \mt|\cite| no final \cite{turing1937}. Essa
é mais verdadeira ainda pois tem um (\mt|\cite{turing1937,dijkstra1968}|) no
final \cite{turing1937,dijkstra1968}. Já esta frase inofensiva usa
\mt|\citeonline{dijkstra1968}| para citar \citeonline{dijkstra1968}
nominalmente. Essa outra frase cita o trabalho que \citeonline{golub1992}
escreveu com outros 9 autores.

A lista abaixo mostra o efeito de \mt|\autoref{}| com capítulos e (sub)seções.

\begin{itemize}
\item \autoref{ch:ex}
\item \autoref{sec:stuff}
\item \autoref{sec:more}
\item \autoref{sec:yet-more}
\end{itemize}

Atenção! O template da BU deixa figuras e tabelas alinhadas à esquerda. No
entanto, o tutorial de Word disponibilizado pela BU diz que legendas e
\emph{captions} devem respeitar o ``alinhamento da ilustração'' (e apresenta
uma ilustração alinhada à esquerda). O tutorial explicando a ABNT mostra uma
figura centralizada com legendas alinhadas a esquerda e com recuo até o começo
da figura. O autor do \texttt{.cls} se exime de qualquer culpa. Alinhe aqui
(com \mt|\centering|, \mt|\flushright| ou \mt|\flushleft|) como mandar o seu
coração.

\section{Coisas}
\label{sec:stuff}
Imagine alguma afirmação de alto valor científico aqui.

\subsection{Outras coisas}
\label{sec:more}
Olá! Eu vim do passado para te avisar que o texto de uma dissertação deve ser
impessoal e você não deveria tentar conversar com o leitor.

\subsubsection{Outras coisas mais}
\label{sec:yet-more}
Estudos demonstram que essa afirmação é falsa.

\subsubsubsection{Ainda outras coisas mais}
\label{sec:yet-another}
Fazer a grama verde, como? Novamente o jogo foi perdido. Opcionalmente, tudo
pode ser opcional. Recursos foram gastos com isso. Descubra a verdade nas
capitalizadas.
% Fiquei 15 minutos mais próximo da morte ao escrever isso. Você pode chegar
% ainda mais perto se tentar entender.


%%%%%%%%%%%%%%%%%%%%%%%%%%%%%%%%%%%%%%%%%%%%%%%%%%%%%%%%%%%%%%%%%%%%
%%% Elementos pós-textuais                                       %%%
%%%%%%%%%%%%%%%%%%%%%%%%%%%%%%%%%%%%%%%%%%%%%%%%%%%%%%%%%%%%%%%%%%%%

\postextual
\bibliography{userguide}

\end{document}
