\documentclass[a4paper]{ufsc-thesis-rn46-2019}  

% Gerador de texto
\usepackage{lipsum}

\usepackage{cmap}
\usepackage[utf8]{inputenc}
\usepackage[T1]{fontenc}

% Preâmbulo
\titulo{Exemplo do uso da class \abnTeX}
\autor{Fulano da Silva}
\data{\today}
\instituicao{Universidade Federal de Santa Catarina}
\local{Florianópolis, SC-Brasil}
\tipotrabalho{Exemplo para referência futura}
\orientador{Pedro Henrique {}de Souza}
\coorientador{José {}da Silva Sauro}
\programa{Programa de Pós-Graduação em Ciências da Computação}
\preambulo{Modelo can\^{o}nico de trabalho de conclusão de curso para acadêmicos da UFSC e usuários da plataforma \abnTeX.}
\centro{Centro Tecnológico -- CTC}
\assuntos{Ciências da Computação,Modelos,Teses,OpenSource,LaTeX}

\begin{document}
% Inicia parte pré-textual do documento capa, folha de rosto, folha de
% aprovação, aprovação, resumo, lista de tabelas, lista de figuras, etc.
\pretextual%
\imprimircapa%
\imprimirfolhaderosto*%
\clearpage
% Deve ser gerado no site http://ficha.bu.ufsc.br/
% \imprimirfichacatalografica%
\tableofcontents%
\textual%
\cleardoublepage

\chapter{Introduction}
This is the introduction. Lorem ipsum dolor sit amet, consectetur adipiscing elit, sed do eiusmod tempor incididunt ut labore et dolore magna aliqua. Ut enim ad minim veniam, quis nostrud exercitation ullamco laboris nisi ut aliquip ex ea commodo consequat. Duis aute irure dolor in reprehenderit in voluptate velit esse cillum dolore eu fugiat nulla pariatur. Excepteur sint occaecat cupidatat non proident, sunt in culpa qui officia deserunt mollit anim id est laborum.

\lipsum[1]

\chapter{Background}

\lipsum[1]

\section{Some stuff}

\lipsum[1]

\section{Some other stuff}

\lipsum[1]

\lipsum[1]

\lipsum[1]

\lipsum[1]

\lipsum[1]

\lipsum[1]

\lipsum[1]


\end{document}
