\documentclass[embeddedlogo]{ufsc-thesis-rn46-2019}

\usepackage[T1]{fontenc} % fontes
\usepackage[utf8]{inputenc} % UTF-8
\usepackage{lipsum} % Gerador de texto
\usepackage{pdfpages} % Inclui PDF externo (ficha catalográfica)

% Usado para mostrar código
\usepackage{minted}
\newmintinline[mt]{latex}{fontsize=\normalsize}
\newmintinline[mft]{latex}{fontsize=\footnotesize}
\setminted{fontsize=\tiny,linenos,xleftmargin=2em}
\setmintedinline{breaklines,breakbytokenanywhere}

% Usado para mostrar comandos latex nesse guia:
\newcommand{\lacmd}[1]{\texttt{\textbackslash{}#1}}
\newcommand{\laenv}[1]{\texttt{\textbackslash{}begin\{#1\}...\textbackslash{}end\{#1\}}}
\newcommand{\laenvi}[2]{\texttt{\textbackslash{}begin\{#1\}[#2]...\textbackslash{}end\{#1\}}}

%%%%%%%%%%%%%%%%%%%%%%%%%%%%%%%%%%%%%%%%%%%%%%%%%%%%%%%%%%%%%%%%%%%%
%%% Configurações da classe (dados do trabalho)                  %%%
%%%%%%%%%%%%%%%%%%%%%%%%%%%%%%%%%%%%%%%%%%%%%%%%%%%%%%%%%%%%%%%%%%%%

% Informações para capa e folha de rosto/certificacao

% Caso o título contenha alguma porção LaTeX ilegível, defina um título
% alternativo opcional com []'s para ser usado no campo Title do PDF
% IMPORTANTE: Os títulos deveriam ser iguais. Apenas use um título
% alternativo se o título não puder ser expresso com letras e números
\titulo[Template \LaTeX{} com metadados]%
       {Template \LaTeX{} seguindo a RN 46/2019/CPG da UFSC}

\autor{Omar Ravenhurst}
\data{1 de agosto de 2019} % ou \today
\instituicao{Universidade Federal de Santa Catarina}
\centro{Centro Tecnológico}
\local{Florianópolis} % Apenas cidade! Sem estado
\programa{Programa de Pós-Graduação em Ciência da Computação}
% Os dois próximos itens são usados para gerar o \preambulo
\tese % ou \dissertacao ou \tcc
\titulode{doutor em Ciência da Computação}

%%% Atenção! No caso de TCC, além de usar \tcc, outros comandos devem ser fornecidos:
%%%
% \tcc
% \departamento{Departamento de Informática e Estatística}
% \curso{Ciência da Computação}
% \titulode{Bacharel em Ciência da Computação}
% %% Para TCCs, orientadores e coorientadores podem ser externos, logo a
% %% BU exige que sua afiliação seja explicitada. Por padrão, assume-se
% %% UFSC. Você pode alterar a afiliação com os comandos abaixo:
% \afiliacaoorientador{Universidade Federal de Santa Catarina}
% \afiliacaocoorientador{Universidade Federal da Terra de Ninguém}

% Orientador, coorientador, membros da banca e coordenador
% As regras da BU agora exigem que Dr. apareça **depois** do nome
% Dica: para gerar Profᵃ. use Prof\textsuperscript{a}.
% Dica 2: para feminino use \orientadora e \coorientadora
\orientador{Prof. Ben Trovato, Dr.}
\coorientador{Prof. Lars Thørväld, Dr.}
\membrabanca{Prof\textsuperscript{a}. Valerie Béranger, Dr.}{Universidade Federal de Santa Catarina}
\membrobanca{Prof. Mordecai Malignatus, Dr.}{Universidade Federal de Santa Catarina}
\membrobanca{Prof. Huifen Chan, Dr.}{Universidade Federal da Terra de Ninguém}
% Dica: se feminino, \coordenadora
\coordenador{Prof. Charles Palmer, Dr}

\begin{document}

%%%%%%%%%%%%%%%%%%%%%%%%%%%%%%%%%%%%%%%%%%%%%%%%%%%%%%%%%%%%%%%%%%%%
%%% Principais elementos pré-textuais                            %%%
%%%%%%%%%%%%%%%%%%%%%%%%%%%%%%%%%%%%%%%%%%%%%%%%%%%%%%%%%%%%%%%%%%%%

% Inicia parte pré-textual do documento capa, folha de rosto, folha de
% aprovação, aprovação, resumo, lista de tabelas, lista de figuras, etc.
\pretextual%
\imprimircapa%
\imprimirfolhaderosto*
% Atenção! \cleardoublepages são inseridos automaticamente
% Atenção! esse \protect é importante
\protect\incluirfichacatalografica{ficha.pdf}
\imprimirfolhadecertificacao

% Listas de "coisas". O * no final faz com que as listas não sejam 
% incluídas como entratas do sumário (\tableofcontents)

% \listoftables*
% \listofalgorithms*
% \listoffigures*
% \tableofcontents*

\begin{dedicatoria}
  Este trabalho é dedicado à wikipedia e ao stackoverflow. 
\end{dedicatoria}

\begin{agradecimentos}
  O presente trabalho foi realizado com apoio da Coordenação de Aperfeiçoamento de Pessoal de Nível Superior -- Brasil (CAPES) -- Código de Financiamento 001.
\end{agradecimentos}

\begin{epigrafe}
  For a number of years I have been familiar with the observation that the quality of programmers is a decreasing function of the density of go to statements in the programs they produce \\
  \cite{dijkstra1968}
\end{epigrafe}


\begin{resumo}[Resumo]
  Aqui deve ser inserido um resumo de 150 a 500 palavras (limitação de tamanho dada pela BU). A linguagem deve ser português e a hifenização já foi alterada. O resumo em português deve preceder o resumo em inglês, mesmo que o trabalho seja escrito em inglês. A BU também diz que deve ser usada a voz ativa e o discurso deve ser na 3ª pessoa. A estrutura do resumo pode ser similar a estrutura usada em artigos: Contexto -- Problema -- Estado da arte -- Solução proposta  -- Resultados.

  % Atenção! a BU exige separação através de ponto (.). Ela recomanda de 3 a 5 keywords
  \vspace{\baselineskip} 
  \textbf{Palavras-chave:} Palavra-chave. Ponto como separador. Bla.
\end{resumo}


\begin{resumo}[Resumo Estendido]
  \section*{Introdução} 
  A hifenização é alterada para \texttt{brazil}, mesmo para documentos em inglês. Descrever brevemente esses itens exigidos pela BU. Como a RN 95/CUn/2017 é mais recente e impõe outras regras a revelia de regimentos e regulamentos, é mais sábio obedecê-la. Lembre que esse resumo estendido deve term entre 2 e 5 páginas.
  
  \lipsum[1]
  \section*{Objetivos} 
  \lipsum[2]
  \section*{Metodologia} 
  \lipsum[3]
  \section*{Resultados e Discussão} 
  \lipsum[4]
  \section*{Considerações Finais} 
  \lipsum[5]

  \vspace{\baselineskip}  % Atenção! manter igual ao resumo
  \textbf{Palavras-chave:} Palavra-chave. Outra Palavra-chave composta. Bla.
\end{resumo}


\begin{abstract}
  Enlish version of the plain ``resumo'' above. Done with environment
  \texttt{abstract}. Hyphenization is automatically changed to english.

  \vspace{\baselineskip} 
  \textbf{Keywords:} Keyword. Another Compound Keyword. Bla.
\end{abstract}

\listoffigures*  % O * evita que apareça no sumário

\begin{listadesimbolos}
  $\gets$   & Atribuição \\
  $\exists$   & Quantificação existencial \\
  $\rightarrow$   & Implicação \\
  $\wedge$   & E lógico \\
  $\vee$   & Ou lógico \\
  $\neg$   & Negação lógica \\
  $\mapsto$   & Mapeia para \\
  $\sqsubseteq$   & Subclasse (em ontologias) \\
  $\subseteq$   & Subconjunto: $\forall x\;.\; x \in A \rightarrow x \in B$ \\
  $\langle\ldots\rangle$ & Tupla \\
  $\forall$   & Quantificação universal \\
  mmmmm & Nenhum sentido, apenas estou aqui para demonstrar a largura máxima dessas colunas. Ao abrir o ambiente \texttt{listadesimbolos}, pode-se fornecer um argumento opcional indicando a largura da coluna da esquerda (o default é de 5em): \texttt{\textbackslash{}begin\{listadesimbolos\}[2cm] .... \textbackslash{}end\{listadesimbolos\}} \\
  $\alpha$   & Alpha \\
  $\beta$   & Beta \\
  $\gamma$   & Gamma \\
  $\delta$   & Delta \\
  $\epsilon$   & Epsilon \\
  $\zeta$   & Zeta \\
  $\eta$   & Eta \\
  $\theta$   & Theta \\
  $\iota$   & Iota \\
  $\kappa$   & Kappa \\
  $\lambda$   & Lambda \\
  $\mu$   & Mu \\
  $\nu$   & Nu \\
  $\xi$   & Xi \\
  $\pi$   & Pi \\
  $\rho$   & Rho \\
  $\sigma$   & Sigma \\
  $\tau$   & Tau \\
  $\upsilon$   & Upsilon \\
  $\phi$   & Phi \\
  $\bowtie$  & Apertem os cintos, uma quebra de página se aproxima! \\
  $\oslash$   & Não use exclamações em lista de símbolos! \\
  $\varphi$   & Varphi \\
  $\chi$   & Chi \\
  $\psi$   & Psi \\
  $\omega$   & Omega \\

\end{listadesimbolos}

\tableofcontents*%

%%%%%%%%%%%%%%%%%%%%%%%%%%%%%%%%%%%%%%%%%%%%%%%%%%%%%%%%%%%%%%%%%%%%
%%% Corpo do texto                                               %%%
%%%%%%%%%%%%%%%%%%%%%%%%%%%%%%%%%%%%%%%%%%%%%%%%%%%%%%%%%%%%%%%%%%%%
\textual%

\chapter{Introdução}

Bem-vindo ao guia de usuário da classe \texttt{ufsc-thesis-rn46-2019}. Essa
classe é um conjunto de customizações aplicadas à classe
\href{https://ctan.org/pkg/abntex2}{\abnTeX} e ao pacote \texttt{abntex2cite}.
O objetivo da classe \texttt{ufsc-thesis-rn46-2019} é simplório: adequar o
\abnTeX{} às \href{http://portal.bu.ufsc.br/normalizacao/}{normas emitidas pela
Biblioteca Universitária da UFSC} em sequência à
\href{https://repositorio.ufsc.br/handle/123456789/197121}{Resolução Normativa
nº 46/2019/CPG}.


\section{Autores, suporte e atualizações}

Essa classe foi escrita inicialmente por dois alunos do
\href{http://ppgcc.posgrad.ufsc.br/}{PPGCC da UFSC}:
\href{mailto:alexishuf@gmail.com}{Alexis Huf} e
\href{mailto:gustavo.zambonin@posgrad.ufsc.br}{Gustavo Zambonin}.  Há o risco de
esse arquivo não ser atualizado a cada \textit{pull request}, então confira a
lista de mártires no GitHub. Essa classe é mantida no repositório
\href{https://github.com/alexishuf/ufsc-thesis-rn46-2019/}{alexishuf/ufsc-thesis-rn46-2019}.
Atualizações podem ser encontradas nesse repositório. \textit{Issues} e PRs são
bem vindos.

\subsection{Registro de mudanças}

Lista de versões (pelo menos das versões que receberam um número):
\begin{description}
\item[v0.2] (2019-10-30) Engloba alterações feitas pela BU na segunda metade de
  setembro, bugfixes, \lacmd{listadesimbolos} e \lacmd{tcc}.
\item[v0.1] (2019-08-02) Primeira versão completa e documentada.
\item[v0.1-alpha] (2019-08-01) Primeira versão para ajudar alunos com pouco
    prazo de entrega.
\end{description}

\section{Guia rápido}
\label{sec:quick}

A classe \texttt{ufsc-thesis-rn46-2019} deveria ser encarada como um
\emph{drop-in} para o \abnTeX. Você pode começar a escrever uma tese do zero
usando apenas esta classe, mas talvez você queira usar algum template indicado
por algum colega. Na maioria dos casos, bastará usar
\lacmd{documentclass\{ufsc-thesis-rn46-2019\}}, ajustando o caminho se necessário.
Há duas principais formas de incluir essa classe no seu projeto.

\begin{enumerate}
  \item Copie o arquivo \texttt{ufsc-thesis-rn46-2019.cls}. Sim, isso funciona
      pois o logo da UFSC está embutido em base64 dentro da classe. Para que
        isso funcione, você deve compilar com \texttt{-shell-escape} em um
        ambiente \textit{UNIX-like}, como no arquivo \texttt{Makefile} de
        exemplo.
  \item Copie a pasta do repositório do GitHub para dentro do seu projeto.
      Nesse caso você deverá adicionar o caminho dessa pasta ao fazer o
        \lacmd{documentclass}. Há 3 formas de incluir essa pasta:
  \begin{enumerate}
    \item Use a opção ``\emph{Download ZIP}'' do GitHub se você não sabe o que
        é Git.
    \item Via \texttt{git submodule}, se você já está usando Git.
    \item Via \texttt{git clone} se não estiver usando Git no seu projeto
        (\emph{shame on you}) ou se estiver usando no Overleaf, onde submódulos
          não funcionam.
    \end{enumerate}
\end{enumerate}

\begin{figure}[tb]
  \centering
  \caption{Preâmbulo de uma tese (ou dissertação) típica usando esta classe.}
  \label{fig:preambulo}

  \begin{minted}{latex}
\documentclass[english]{ufsc-thesis-rn46-2019/ufsc-thesis-rn46-2019}

% \usepackage's a gosto

\titulo{Template \LaTeX~ seguindo a RN 46/2019/CPG da UFSC}
\autor{Fulano da Silva}
\data{1 de Agosto de 2019}
\instituicao{Universidade Federal de Santa Catarina}
\centro{Centro Tecnológico}
\programa{Programa de Pós-Graduação em Ciências da Computação}
\local{Florianópolis} % Apenas cidade! Sem estado
\tese % ou \dissertacao
\titulode{Doutor em Ciência da Computação}
\orientador{Prof. Dr. Ben Trovato}
\coorientador{Prof. Dr. Lars Thørväld}

\membrobanca{Prof. Valerie Béranger, Dr.}{Universidade Federal de Santa Catarina}
\membrobanca{Prof. Mordecai Malignatus, Dr.}{Universidade Federal de Santa Catarina}
\membrobanca{Prof. Huifen Chan, Dr.}{Universidade Federal de Santa Catarina}
\coordenador{Prof. Dr. Charles Palmer}
  \end{minted}
  \fonte{o autor.}
\end{figure}

Esse guia rápido assume que você está começando do zero, sem um template
anterior, com intuito de ser didático. Se o seu trabalho for escrito em língua
inglesa, você deverá passar uma opção \texttt{english} para a classe. Caso
contrário não precisa passar nenhuma opção. Logo após chamar a classe, você
deverá fornecer dados para a classe (e para o \abnTeX). Veja um preâmbulo
completo de exemplo na \autoref{fig:preambulo}. Os próximos parágrafos mostram um
rápido \emph{overview} sobre o significado de cada um dos comandos usados (a
maior parte deles continua funcionando como no \abnTeX).

\begin{itemize}
  \item \lacmd{titulode}, \lacmd{autor}, \lacmd{instituicao}, \lacmd{orientador}
e \lacmd{coorientador}: esses comandos continuam com o mesmo significado e uso
que possuem no \abnTeX. Para facilitar o uso, também foram adicionados
\lacmd{orientadora} e \lacmd{coorientadora}.
  \item \lacmd{data}: as regras da UFSC exigem que apenas o ano esteja presente
na capa e folha de rosto. No entanto, a nova folha de certificação, que é gerada
por essa classe precisa da data completa. Forneça a data completa \textbf{em
português} (mesmo para documentos em inglês). A classe irá extrair o ano.
  \item \lacmd{programa} e \lacmd{centro}: nome do Programa de Pós-Graduação,
por extenso, e nome do centro, \emph{e.g.}  Centro Tecnológico.
  \item \lacmd{tese} e \lacmd{dissertacao}: o texto a ser colocado abaixo do
título na folha de rosto tem regras bem definidas. Você deve usar um desses dois
comandos para indicar o tipo de trabalho (e consequentemente o nível).
Evidentemente, são mutuamente exclusivos.
  \item \lacmd{preambulo}: Esse comando fornecido pelo \abnTeX{} é desnecessário
nessa classe. Entretanto, você pode usar ele para sobrescrever o texto gerado
automaticamente a partir do uso de \lacmd{tese} e \lacmd{titulode}.
\end{itemize}

A ``folha de certificação'', que substituiu a antiga folha de aprovação, agora
é gerada em \LaTeX. O nome do orientador já foi fornecido anteriormente, resta
apenas indicar os membros avaliadores da banca e o coordenador do programa:
\begin{itemize}
  \item \lacmd{membrobanca{<nome>}{<instituição>}}: adiciona um membro da banca.
\textbf{Atenção}: para membros avaliadores, o ``Dr.'' deve ser inserido após o
nome e o ``Prof(a).'' deve preceder o nome.
  \item \lacmd{coordenador} e \lacmd{coordenadora}: Configura o nome do(a)
coordenador(a) do Programa de Pós-Graduação.
\end{itemize}

Os elementos pré-textuais permanecem em grande parte tendo seu comportamento
determinado pelo \abnTeX. Alguns elementos incluídos nesse guia foram alterados
para satisfazer normas da UFSC. Esses elementos também presentes nesse guia são
obtidos com o código mostrado na \autoref{fig:pre}. Capa (\lacmd{imprimircapa}),
folha de rosto (\lacmd{imprimirfolhaderosto*}) listas (\emph{e.g.}
\lacmd{listoffigures*}) e o sumário \lacmd{tableofcontents*} são obtidos através
dos mesmos comandos do \abnTeX, mas agora respeitam novas regras definidas pela
BU.

\begin{figure}[tb]
  \centering
  \caption{Elementos pré-textuais.}
  \label{fig:pre}
  \begin{minted}{latex}
\pretextual%
\imprimircapa%
\imprimirfolhaderosto*
\protect\incluirfichacatalografica{ficha.pdf}
\imprimirfolhadecertificacao

\begin{dedicatoria}...\end{dedicatoria}  % opcional
\begin{agradecimentos}...\end{agradecimentos} % opcional
\begin{epigrafe}...\end{epigrafe} % opcional
\begin{resumo}...\end{resumo} 

\begin{resumo}[Resumo Estendido] % opcional se trabalho for em português
  \section*{Introdução}
  \section*{Objetivos} 
  \section*{Metodologia} 
  \section*{Resultados e Discussão} 
  \section*{Considerações Finais} 

  % Keywords devem devem ser iguais (mas traduzidas) nos 3 resumos  
  \vspace{\baselineskip}
  \textbf{Palavras-chave:} Palavra 1. Palavra 2. Palavra 3. 
\end{resumo}

\begin{abstract}...\end{abstract}
\listoffigures* % primeiro \listof na ABNT
\listoftables*  
\listofalgorithms*
% sugestão: adicionar outros \listof* aqui
\begin{siglas}...\end{siglas} % há pacotes alternativos que podem agilizar
\tableofcontents*  %deve ser o último
  \end{minted}
  \fonte{o autor.}
\end{figure}

A ficha catalográfica deve ser gerada em sistema próprio da BU, que produz um
PDF. Esse PDF deve ser salvo no seu projeto e incluído no documento.

A macro \lacmd{incluirfichacatalografica{<pdf>}} faz essa inclusão usando o
pacote \href{https://www.ctan.org/pkg/pdfpages}{\texttt{pdfpages}}. O
\lacmd{protect} é necessário para evitar um \emph{bug} (uma linha em branco
substitui o \lacmd{protect}).  A folha de certificação é gerada pelo comando
\lacmd{imprimirfolhadecertificacao} de acordo (e na ordem) com os comandos de
dados emitidos anteriormente.

Antes de iniciar o primeiro capítulo é necessário chamar a macro
\lacmd{textual}, como em todo documento \abnTeX. Após os elementos textuais, os
pós-textuais devem ser precedidos por um \lacmd{postextual}. Para fazer uso da
bibliografia, basta fazer como mostra a \autoref{fig:pos}. A classe já inclui o
\texttt{abntex2cite} e o configura. A UFSC impôs poucas alterações no tratamento
de bibliografia.

\begin{figure}[tb]
  \centering
  \caption{Elementos pós-textuais.}
  \label{fig:pos}
  \begin{minted}{latex}
\postextual
\bibliography{example}

% Atenção! todos os elementos a seguir são opcionais

\apendices
\chapter{Exemplo de Apêndice}
...

\cleardoublepage \printnoidxglossary % Exemplo do pacote glossaries)
\cleardoublepage \printindex % Índice remissívo, e.g., makeidx)
  \end{minted}
  \fonte{o autor.}
\end{figure}

\subsection{TCCs}
\label{sec:tcc}

A BU também definiu regras para Trabalhos de Conclusão de Curso (TCCs). As
diferenças para a pós-graduação estão na capa, folha de rosto e folha de
aprovação (não é feita assinatura digital dos TCCs). Para criar um TCC usando
esse template, você deverá utilizar o comando \lacmd{tcc} no lugar de
\lacmd{tese} ou \lacmd{dissertacao}. No modo TCC, será necessário fornecer
alguns comandos de dados adicionais, como apresentado na
\autoref{fig:tccs-conf}.

No modo TCC, a capa não inclui o brasão da UFSC, mas inclui o nome do
departamento e do curso. Na folha de rosto, a única alteração ocorre no
preâmbulo (abaixo do título). No caso da folha de aprovação as mudanças são
múltiplas. Tanto o Orientador quanto o Coorientador devem ser inseridos na
condição de membros da banca examinadora. Além disso, todos os membros da banca
e o coordenador do curso devem ser apresentados como se assinassem a folha de
aprovação (com linhas horizontais). Orientador e Coorientador precisam ter sua
afiliação explicitada e os demais membros da banca devem ser listados com o
papel de Avaliador (o que não ocorre na folha de certificação).

\begin{figure}[tb]
  \centering
  \caption{Configuração para TCCs.}
  \label{fig:tccs-conf}
\begin{minted}{latex}
\tcc
\departamento{Departamento de Informática e Estatística}
\curso{Ciência da Computação}
\titulode{Bacharel em Ciência da Computação}
\afiliacaoorientador{Universidade Federal de Santa Catarina} %default
\afiliacaocoorientador{Universidade Federal da Terra de Ninguém}
\end{minted}
  \fonte{o autor.}
\end{figure}





\chapter{Problemas conhecidos}
\label{ch:problems}

\textbf{Atenção!} Há regras de formatação que não podem ser garantidas por essa
classe (ainda) devido a detalhes técnicos ou devido à forma como o \LaTeX e
BibTeX funcionam.

\section{Ordem de Elementos pré- e pós-textuais}
As regras da BU não alteram (nem alteravam antes) o significado e a ordem de
elementos pré-textuais definidos pela ABNT. Essa classe não tem como ajudá-lo.

\section{Alinhamento de Figuras}
O alinhamento de Figuras (ainda) não está totalmente claro nas regras da BU. Os
documentos da BU apresentam duas formas distintas. O autor do documento é
encorajado a continuar usando o alinhamento que usava até então. Essa dúvida
deverá ser esclarecida pela BU em breve, ocasião na qual essa classe será
atualizada.

\section{URLs de referências em inglês}

Alguns alunos costumam trocar os dizeres ``Disponível em'' gerados pelo
\texttt{abntex2cite} por ``Available at'' quando redigem o texto em inglês. No
entanto \textbf{isso é errado}! De acordo com o suporte da BU, apenas o corpo do
texto está em inglês, e assim como vários elementos pré-textuais permanecem em
português, os dizeres Disponível em deveriam permanecer em português.

Caso você deseje \textbf{contrariar a orientação acima}, a forma mais segura (no
sentido do LaTeX, não no sentido da entrega na BU) de \textbf{violá-la} é usar o
campo \texttt{note} do BibTeX. Você deverá fazer o equivalente a
\texttt{Available at: \lacmd{url}\{https://doi.org/10.1145/3338112\}} em cada
entrada no seu arquivo \texttt{.bib}, além de desativar o campo \texttt{url}
removendo-o ou renomeando-o para. \texttt{x-url}.

\chapter{Referência de opções e macros}
\label{ch:ref}

\section{Opções da classe}

A classe possui apenas uma opção que fornece recursos: a primeira na lista
abaixo. Todas as demais servem para desativar comportamentos padrão.

\paragraph*{\texttt{english}} Prepara o \abnTeX{} e a classe para que o corpo
do documento esteja em inglês. Alguns elementos em português, especialmente os
relacionados a BU e a UFSC permanecem em português.

\paragraph*{\texttt{brazil}} Não tem efeito pois o idioma por padrão é o
português do Brasil. Entretanto, note que para utilizar outros idiomas além de
inglês e português, é necessário repassar todos os idiomas explicitamente à
classe \abnTeX.

\paragraph*{\texttt{oneside}} Essa opção é destinada à classe \texttt{memoir} (a
classe \abnTeX) é uma extensão do \texttt{memoir}. Ela fará com que todas as
páginas sejam tratadas como anverso, eliminando a alternância na largura das
margens esquerda e direita. Essa opção também \textbf{elimina a inserção de
  páginas em branco}. Embora isso viole as instruções da BU, é uma \emph{feature} útil
para alunos apresentando relatórios parciais, como EQM, SAD e EQD. Alguns
programas, como o PPGCC impõem limites nominais à quantidade de páginas.

\paragraph*{\texttt{coorientadorbanca}}
No caso de documentos de TCC (e apenas nesse caso), essa opção faz com que o(s)
coorientadore(s) sejam listados como membros da banca na ``Folha de aprovação''
que substitui a ``Folha de Certificação''. 

\paragraph*{\texttt{nopageanchorhack}}
Como a capa é a página 0, \lacmd{imprimircapa} faz
\lacmd{hypersetup\{pageanchor=false\}} e isso é desfeito em instruções
posteriores. Isso é um
\href{https://tex.stackexchange.com/a/331766}{\emph{workaround}} para evitar um
warning inofensivo, mas recorrente sobre âcoras terem sido sido definidas
multiplas vezes para a mesma página:
\begin{minted}{text}
pdfTeX warning (ext4): destination with the same identifier (name{page.1})
has been already used, duplicate ignored
\end{minted}
Caso seu trabalho faça \lacmd{setcounter\{page\}\{1\}} após a capa, esse
\emph{workaround} perderá o efeito. Essa opção instrui a classe a não aplicar o
workaround, para que você o faça:
\begin{itemize}
\item Adicione \lacmd{hypersetup\{pageanchor=false\}} antes de \lacmd{imprimircapa}
\item Adicione \lacmd{hypersetup\{pageanchor=\textbf{true}\}} após o
      \lacmd{setcounter\{page\}\{1\}}.
\end{itemize}

\paragraph*{\texttt{embeddedlogo}} Se ativada, essa opção desempacota o logo da
UFSC diretamente de um base64 dentro do arquivo \texttt{.cls}. Isso permite que
o \texttt{.cls} seja um arquivo auto-contido. Pessoas avessas a \texttt{git
  submodule} e a diretórios podem simplesmente levar o \texttt{.cls} de um
projeto para outro. Embora versátil, essa opção não é padrão pois exige que a
compilação seja realizada com a opção \texttt{-shell-escape}. O logo foi obtido
da \href{http://identidade.ufsc.br/}{página de identidade visual da UFSC}. 

\paragraph*{\texttt{unix}} A opção \texttt{embeddedlogo} implica em uam
tentativa de auto-detecção de sistema operacional para determinar se deveria ser
utilizado o comando \texttt{base64} (disponível em sistemas UNIX) ou o comando
\texttt{certutil} que existe em sistemas Windows. A opção unix desativa a
auto-detecção e cosnidera que o sistema é um sistema UNIX-like (linux, Mac OS ou
cygwin).

\paragraph*{\texttt{windows}} Mesmo efeito que \texttt{unix}, mas faz com que o
sistema seja considerado Windows.

{\sloppy
\paragraph*{\texttt{logodir=OUTDIR}} Em combinação com \texttt{embeddedlogo},
gera os arquivos \lacmd{jobname.64} e \lacmd{jobname-logo.pdf} dentro de
\emph{OUTDIR}. \textbf{Importante}: \emph{OUTDIR} precisa existir antes da
execução do \texttt{pdflatex}. A classe não criará o diretório
\par}

\paragraph*{\texttt{latexmkoutdir}} Quando a opção \texttt{-outdir=OUT} do
\texttt{latexmk} é utilizada (com \texttt{-cd} ou \texttt{-cd-}), o ambiente
\texttt{filecontents} irá gerar arquivos em \texttt{\emph{OUTDIR}}. No entanto o
comportamento de \lacmd{write18} não será afetado. Isso irá se manifestar na forma de
um erro similar a ``\texttt{base64: thesis.64: No such file or directory}'',
seguido de uma falha de compilação por não encontrar o arquivo
\lacmd{jobname-logo.pdf} (\lacmd{jobname} expande para \texttt{main} quando o
arquivo compilado se chama \texttt{main.tex}). Essa opção informa a classe que a
compilação está ocorrendo via \emph{latexmk}. Note que para que a compilação
funcione, \textbf{também é necessário} fornecer a opção
\texttt{logodir=\emph{OUTDIR}} onde \texttt{\emph{OUTDIR}} é o mesmo fornecido
para a opção \texttt{-outdir} do \texttt{latexmk}

\paragraph*{\texttt{times}, \texttt{lmodern}, \texttt{arial}} Escolhe a fonte
do documento. O padrão é \texttt{times}. No caso das escolhas \texttt{times} e
\texttt{arial} são usados clones \emph{open-source} (sem limitações legais à
redistribuição) que podem ser encontradas em qualquer instalação \LaTeX. A
opção \texttt{lmodern} corresponde à fonte Latin Modern, a fonte padrão do
\LaTeX. Essa é uma fonte serifada similar à Times, mas não possui respaldo
oficial da BU.

\paragraph*{\texttt{ilustracoes}} Restaura o nome da ``Lista de Figuras'' para
``Lista de Ilustrações'' (default do abntex2). ``Ilustrações'' é usado na NBR
14724:2011 e nos slides/tutorias da BU, enquanto os templates da BU usam
``Figuras''. Consultada, a BU declarou que ilustrações é usado apenas como termo
genérico (cobrindo figuras, quadros, tabelas, etc.) e que se a lista só contém
figuras deveria ser chamada de ``Lista de Figuras''. Como listar
não-\texttt{figure}s em \lacmd{listoffigures} é um cenário extrmamente exótico
em \LaTeX, essa classe utiliza ``Lista de Figuras'' em portugês.

\paragraph*{\texttt{nocapautoref}} Essa classe configura o comando
\lacmd{autoref} para que nomes de capítulos e seções sejam capitalizados. Ao
fornecer essa opção essa alteração deixará de ser feita.

\paragraph*{\texttt{noabntexcite}} Essa classe inclui o pacote
\texttt{abntex2cite} e o configura automaticamente. Caso seja necessário fazer
essa inclusão manualmente para fazer sua própria configuração, use essa opção
para que a classe não carregue o pacote.

\paragraph*{\texttt{nohidelinks}} O
\href{https://ctan.org/pkg/hyperref}{\texttt{hyperref}} por padrão desenha
bordas coloridas ao redor de \emph{hyperlinks}. Por padrão a classe desativa
esse comportamento, passando a opção \texttt{hidelinks} ao \texttt{hyperref}.
Ao ativar essa opção o \texttt{hyperref} voltará ao seu próprio comportamento
padrão.

\paragraph*{\texttt{noplainurl}} A norma ABNT NBR6023:2018 não utiliza mais URLs
na forma <http://example.org>, e apresenta a URL diretamente, sem os <>.  Esse
comportamento sempre é aplicado (a menos que seja usada a opção
\texttt{noabntexcite}). Para uniformizar o estilo, essa classe aplica a mesma
configuração no corpo do documento (nas instâncias onde se usaria
\lacmd{url\{http://example.org\}}. Ao fornecer a opção \texttt{noplainurl} as
configurações do comando \lacmd{url} não são alteradas.


\paragraph*{\texttt{nocleanheader}} O \abnTeX{} gera cabeçalhos na página que
incluem o título da seção. Essa classe suprime esse estilo de página,
proveniente da classe base \texttt{memoir}, pois eles não existem no template
da BU. Ao passar essa opção o cabeçalho padrão do \abnTeX{} será restaurado.

\paragraph*{\texttt{nopretextualbookmark}} O \abnTeX{} gera \emph{bookmarks} no
PDF para cada elemento pré-textual, exceto sumário, \lacmd{listoffigures} e
similares. Esses \emph{bookmarks} são o que alimenta o ``table of contents'' que
os leitores PDF apresentam, usualmente em uma barra lateral.  Ao fornecer essa
opção, o \abnTeX{} deixará de criar esses \emph{bookmarks} para os elementos
pré-textuais (e continuará criando para os elementos textuais e pós-textuais.

\paragraph*{\texttt{nocleardoublepage}} Por padrão, os comandos
\lacmd{tableofcontents} \lacmd{listoftables}, \lacmd{listoffigures},
\lacmd{listoflistings} (se estiver definido) são modificados logo após o
\lacmd{begin\{document\}} para que realizem um \lacmd{cleardoublepage}. Logo,
esseselementos serão impressos garantidamente em uma nóva página que ficará no
anverso. Essa opção, se presente, desativa a injeção dessa alteração

\paragraph*{\texttt{nolatexextra}}
Se fornecida, essa opção impede o \lacmd{RequirePackage} de pacotes em \texttt{latexextra}.
Tais pacotes estão disponíveis no overleaf e em práticamente qualquer sistema
\LaTeX. Caso instalar tais pacotes naõ seja possível, essa opção pode ser usada.
No entanto, outros pacotes, que o seu documento usa diretamente provavelmente
continuarão dependendo desses pacotes.

\paragraph*{\texttt{nopdfinfo}} Por padrão, os metadados definidos pelos
comandos \lacmd{titulo} \lacmd{autor} serão atribuídos aos campos de metadados
\texttt{Title} e \texttt{Author} do PDF através das opções \texttt{pdftitle} e
\texttt{pdfauthor} do \texttt{hyperref}. Essa opção se presente impede isso.
\textbf{Cuidado}: Esses metadados não tem relação com o PDF/A, exigido pela BU.

\paragraph*{\texttt{*}} Quaisquer outras opções não listadas anteriormente são
repassadas à classe \abnTeX{} sem manipulação.

\section{Novas Macros}

Essa seção apresenta macros novas, não presentes no \abnTeX{} ou
\texttt{abntex2cite}.

\subsection{Comandos de dados}
\label{sec:dados}

Esses comandos visam fornecer dados que são posteriormente usados pelos
comandos de saída discutidos na próxima seção. \textbf{Importante:} não esqueça
de preencher os comandos de dados do próprio \abnTeX, como mostrado na
\autoref{sec:quick}.

\paragraph*{\lacmd{programa\{<nome>\}}} Configura o nome do Programa de
Pós-Graduação.

\paragraph*{\lacmd{programacotutela\{<nome>\}}} Tradução a ser usada no lugar de
\lacmd{\programa} na folha de rosto de cotutela em lingua estrangeira. Veja
\lacmd{\imprimirfolhaderostocotutela}.

\paragraph*{\lacmd{titulode\{<texto>\}}} Define o título alvo.  O título
\textit{deve ser escrito incluindo a especialidade}. Por exemplo, \emph{Mestre
  em Ciência da Computação}.

\paragraph*{\lacmd{titulodecotutela\{<texto>\}}} Uma tradução para
\lacmd{titulode} a ser usada na folha de rosto em língua estrangeira (veja
\lacmd{imprimirfolhaderostocotutela}).

\paragraph*{\lacmd{tese}} Comando sem argumentos. Marca o documento como uma
tese de doutorado. Isso irá alterar o preâmbulo e a folha de certificação.

\paragraph*{\lacmd{dissertacao}} Comando sem argumentos. Marca o documento como
uma dissertação de mestrado. Esse é o padrão quando nem \lacmd{dissertacao} nem
\lacmd{tese} são usados.

\paragraph*{\lacmd{tcc}} Comando sem argumentos. Marca o documento como um
TCC. As diferenças nesse modo são discutidas na \autoref{sec:tcc}.

\paragraph*{\lacmd{departamento\{<departamento>\}}} Informa o departamento ao
qual o curso de graduação ou especialização está vinculado. O Departamento é
incluído na capa.

\paragraph*{\lacmd{curso\{<curso>\}}} Informa o curso de graduação
ou especialização do trabalho. Essa informação é incluída na parte superior da
capa, na folha de rosto e na folha de aprovação.

\paragraph*{\lacmd{cursocotutela\{<texto>\}}} Uma tradução para
\lacmd{curso} a ser usada na folha de rosto em língua estrangeira (veja
\lacmd{imprimirfolhaderostocotutela}).

\paragraph*{\lacmd{nivel\{<nivel>\}}} Deve ser o nível do título almejado, em
minúsculas. O valor padrão é inferido a partir do uso (ou não uso) de
\lacmd{tese} e \lacmd{dissertacao}.

\paragraph*{\lacmd{centro\{<texto>\}}} Centro da UFSC onde está sediado o
Programa de Pós-Graduação.

\paragraph*{\lacmd{centrocotutela\{<texto>\}}} Uma tradução para
\lacmd{centro} a ser usada na folha de rosto em língua estrangeira (veja
\lacmd{imprimirfolhaderostocotutela}).

\paragraph*{\lacmd{data\{<data por extenso>\}}} Define a data a ser usada na
folha de certificação. O ano dessa data será usado na capa e na folha de
rosto. Se esse comando for omitido, a data atual do sistema será utilizada. Não
utilize \lacmd{today} como argumento em documentos em inglês. Isso fará com que
a data da folha de aprovação (que é em português) apareça em inglês! Exemplo:
\lacmd{data\{1 de Agosto de 2019\}}.

\paragraph*{\lacmd{ano\{<ano>\}} \\
            \lacmd{mes\{<nome do mes>\}} \\
            \lacmd{dia\{<dia>\}}}
Define cada componente da data individualmente.

\paragraph*{\lacmd{membrobanca[<papel>]\{<nome>\}\{<universidade>\}} \\
            \lacmd{membrabanca\{<nome>\}\{<universidade>\}}}
Nome do professor avaliador da banca, incluíndo Prof. (ou
Prof\textsuperscript{a}.) antes do nome e o título \textbf{após} o
nome. Exemplo: \emph{Prof. Mordecai Malignatus, Dr.}. O orientador não deve ser
incluído nessa seção, nem mesmo para TCCs. Não está claro nas instruções da UFSC
se o coorientador deveria ser incluído no caso de trabalhos de pós-graduação. O
argumento opcional \texttt{<papel>}, cujo valor padrão é \texttt{Avaliador} só é
utilizado em TCCs (cf. \autoref{sec:tcc}). No caso de TCCs com mulheres na banca
o papel deve ser alterado para \texttt{Avaliadora}, ou deve ser utilizado o
comando \lacmd{membrabanca}

\paragraph*{\lacmd{coordenadora\{<nome>\}}} Nome do coordenador do programa ou
do curso, no caso de TCCs. No caso de coordenadora, o comando
\lacmd{coordenadora} pode ser usado. Exemplo: \lacmd{coordenador\{
  Prof. Mordecai Malignatus, Dr.\}}.

\paragraph*{\lacmd{instituicaocotutela\{<texto>\}}} Uma tradução para
\lacmd{instituicao} a ser usada na folha de rosto em língua estrangeira (veja
\lacmd{imprimirfolhaderostocotutela}). Note que pelo modelo de cotutela da UFSC,
não deve ser usado o nome da instituição estrangeira, mas sim uma o próprio nome
da UFSC. Não está claro se o nome da UFSC deveria ou não ser traduzido. Caso
esse comando não seja invocado, o nome da UFSC será usado sem tradução, como
definido em \lacmd{instituicao}.

\paragraph*{\lacmd{localcotutela\{<texto>\}}} Local da universidade (não a UFSC)
com a qual o aluno tem acordo de cotutela. Não há regra sobre se deveria ser
usado a cidade da sede, do campus ou do departamento. Consulte seu acordo de
cotutela.

\paragraph*{\lacmd{tipotrabalhocotutela\{<texto>\}}} Uma tradução para o
\lacmd{tipotrabalho} a ser usada a ser usada na folha de rosto em língua
estrangeira (veja \lacmd{imprimirfolhaderostocotutela}). Note que se forem
utilizadas as macros \lacmd{tese}, \lacmd{dissertacao} ou \lacmd{\tcc}, será
automaticamente gerada uma tradução do termo em portugês para o inglês.

\paragraph*{\lacmd{orientadorext[<sigla afiliação>]\{<Prof. Nome, Dr.>\}\{<afiliação>\}}}
Define um orientador com o nome e afiliação dados. Nomes de orientadores,
coorientadores e membros de bancas devem ter os títulos (como Dr.) no final e
Prof. (se for um professor) no início. A sigla da afiliação é usada apenas na
folha de rosto quando o trabalho é em cotutela. A afiliação completa é usada na
folha de certificação. \textbf{Importante:} Para professores da UFSC, não
forneça afiliação (inclusive pode-se usar o comando \lacmd{orientador\{Prof.
Nome, Dr.\}}). Fornecer uma afiliação da UFSC fará com que ela seja exibida
onde regras da BU dizem que ela não deveria.

Para definir mais de um orientador, use essa macro ou \lacmd{orientador} mais de
uma vez. A ordem das macros define a ordem dos orientadores. O primeiro
orientador deve ser afiliado à UFSC e apenas ele aparecerá na folha de
certificação (pós graduação). No caso de cotutela na graduação, todos os
orientadores aparecem na folha de aprovação. Lembre que só há mais de um
orientador no caso de cotutela, o que exige
\lacmd{imprimirfolhaderostocotutela}. A \autoref{fig:cotutela} mostra um exemplo
com dois orientadores e dois corientadoras.

\begin{figure}[tb]
  \centering
\begin{minted}{latex}
\orientador{Prof. Fulano Silva, Dr.} % afiliado a UFSC
\orientadoraext[UPAR]{Prof\textsuperscript{a}. François Exemple, Dr.}{Université Paris}
\coorientadora{Prof\textsuperscript{a}. Ciclana Silveira, Dra.}
\coorientadoraext[UFRGS]{Prof\textsuperscript{a}. Josefina Exemplo, Dra.}{Univer}

% .... 
\imprimirfolhaderostocotutela
% - Usará (UFSC) e (UPAR) para os orientadores
% - Usará Orientadores: e Coorientadoras:
% - Em inglês usará Supervisors: e Co-supervisors:
\end{minted}
  \caption{Exemplo cotutela}
  \label{fig:cotutela}
\end{figure}

\paragraph*{\lacmd{orientadoraext[<sigla afiliação>]\{<Prof. Nome, Dr.>\}\{<afiliação>\}}}
Versão feminina de \lacmd{orientadorext}.

\paragraph*{\lacmd{coorientadorext[<sigla afiliação>]\{<Prof. Nome, Dr.>\}\{<afiliação>\}}}
Adiciona um coorientador com uma afiliação. As regras para afiliação são as mesmas
detalhadas em \lacmd{orientadorext}.

\paragraph*{\lacmd{coorientadoraext[<sigla afiliação>]\{<Prof. Nome, Dr.>\}\{<afiliação>\}}}
Versão feminina de \lacmd{coorientadorext}.

\paragraph*{\lacmd{orientadora\{Prof. Nome, Dr.\}}} Versão feminina de \lacmd{orientador}.

\paragraph*{\lacmd{coorientadora\{Prof. Nome, Dr.\}}} Versão feminina de \lacmd{coorientador}.

\subsection{Comandos de saída}

Esses comandos produzem resultados visíveis. Devem ser usados dentro do
ambiente \texttt{document}.

\paragraph*{\lacmd{incluirfichacatalografica\{<arquivo>\}}}
Inclui a ficha catalográfica proveniente de \texttt{arquivo} que foi gerado em
\href{http://ficha.bu.ufsc.br/}{ficha.bu.ufsc.br}.

\paragraph*{\laenv{listadesimbolos}}
Define uma lista de símbolos. O ambiente pode receber um argumento opcional que
especifica a largura da coluna esquerda (onde devem ser inseridos os símbolos. O
valor padrão dessa largura é de \texttt{5em}, o suficiente para conter mmmmm. A
largura da coluna da direita é computada automaticamente de modo que toda a
tabela respeite as margens. Exemplo:

\begin{minted}{latex}
\begin{listadesimbolos}[2cm]
  $\langle\ldots\rangle$ & Tupla \\
  $\forall$   & Quantificação universal \\
  % ....
\end{listadesimbolos}
\end{minted}


\paragraph*{\lacmd{imprimirfolhaderostocotutela[idioma]}} Chama
\lacmd{imprimirfolhaderosto} e imediatamente gera uma folha de rosto com os
dizeres traduzidos para o \texttt{idioma}. Como esse comando gera duas páginas
(usando o anverso e o verso), não existe versão equivalente a
\lacmd{\imprimirfolhaderosto*}. O \texttt{idioma} deve ser um idioma conhecido
pelo \texttt{babel}, como \texttt{english}. Caso o idioma não seja
\texttt{english}, pode ser necessário adicionar uma tradução para alguns termos,
trocando \lacmd{captionsbrazil} por \lacmd{captions\emph{idioma}} no trecho de
código exibido na \autoref{fig:rotulos}, que deve ser inserido no preambulo
antes do \lacmd{\imprimirfolhaderostocotutela}. Termos dinâmicos, como
instituição, curso, título obtido, tipo de trabalho podem ser configurados
utilizando variantes terminando em cotutela dos comandos de dados que os
configuram, como \lacmd{programacotutela} para o \lacmd{programa}.


\begin{figure}[tb]
  \centering
\begin{minted}{latex}
\addto\captionsbrazil{
  \renewcommand{\ufscthesisand}{e}%
  \renewcommand{\orientadorname}{Orientadora:}%
  \renewcommand{\coorientadorname}{Coorientadora:}%
  \renewcommand{\orientadoraname}{Orientadora:}%
  \renewcommand{\coorientadoraname}{Coorientadora:}%
  \renewcommand{\orientadorasname}{Orientadoras:}%
  \renewcommand{\coorientadorasname}{Coorientadoras:}%
  \renewcommand{\orientadoresname}{Orientadores:}%
  \renewcommand{\coorientadoresname}{Coorientadores:}%
}
\end{minted}
  \caption{Tradução rótulos de orientadores e coorientadores}
  \label{fig:rotulos}
\end{figure}

\subsection{Comandos \abnTeX{} modificados}

\paragraph*{\lacmd{titulo[Nos metadados]\{No documento\}}} Além de definir o título é
possível definir uma redação alternativa do título para ser usada nos metadados
do PDF (caso \texttt{nopdfinfo} não tenha sido fornecido como opção da classe).

\paragraph*{\lacmd{imprimircapa}} O modo de uso desse comando permanece o
mesmo. No entanto, como as novas capas devem incluir o brasão da UFSC, há alguns
detalhes a serem observados. Ao utilizar a opção \texttt{embeddedlogo}, o logo
da UFSC é extraíde de um base64 embutido dentro do próprio \texttt{.cls}. Sem
usar essa opção é necessário que o arquivo \texttt{logo-ufsc.pdf} seja
encontrado pelo comando \lacmd{includegraphics}. Para que isso funcione, o
\texttt{.cls} inicializa o \lacmd{graphicspath} da seguinte forma:

\begin{minted}{latex}
  \graphicspath{%
    {.}%
    {ufsc-thesis-rn46-2019/}%
    {../ufsc-thesis-rn46-2019/}%
    {ufsc-thesis/}%
    {../ufsc-thesis/}%
\end{minted}

Essa inicialização é suficiente para os casos de uso típicos. Note que os
caminhos são relativos ao \emph{working directory} do \texttt{pdflatex}, não são
relativos ao \texttt{.cls} ou ao seu \texttt{.tex}.

\paragraph*{\lacmd{imprimirfolhaderosto}} A folha de rosto foi modifica de
acordo com as intruções da BU. O elementa com maior diferênça é o texto do
preambulo (que é inserido abaixo do título). Nessa classe, o texto é gerado
automáticamente a partir da informação provida pelo uso de \lacmd{tese} (ou
\lacmd{dissertacao}), \lacmd{programa} e \lacmd{titulode}, apresentados
anteriormente. Ainda é possível usar \lacmd{preambulo} diretamente para
sobreescrever o texto gerado. No entanto, leia as regras da BU com atenção.

\paragraph*{\laenv{dedicatoria}}
O ambiente \texttt{dedicatoria} do \abnTeX não aplica nenhuma formatação. O
ambiente foi alterado para que o texto fique alinhado a direita, com recuo de
5cm, ancorado na parte inferior da página. Essa formatação se originou do
documento da BU que explica as normas ABNT. A medida de 5cm foi tomada do
template \texttt{.doc} disponibilizado pela BU.

\paragraph*{\laenv{epigrafe}}
O ambiente \texttt{epigrafe} do \abnTeX não aplica nenhuma formatação. O
ambiente foi alterado para que o texto fique alinhado a direita, com recuo de
4cm, espaçamento entre linhas simples e fonte 10. Essas são as mesmas
configurações do ambiente citação. o template \texttt{.doc} aplica todas, exceto
pela fonte 11, apesar de citar a ABNT NBR 10520. Visando maior consistência,
optou-se por usar a fonte 10pt.

\paragraph*{\laenvi{abstract}{<nome>}}
Em abnTeX2, não se deve chamar o ambiente abstract diretamente. Caso o alune o
chame, será interpretado como se o aluno tivesse chamado
\lacmd{begin\{resumo\}[Abstract]} e a linguagem (controlando hifenização) será
alterada para inglês.

\paragraph*{\laenvi{resumo}{<nome>}}
Ambiente usado para gerar os resumos (para documentos em inglês é necessário um
resumo estendido) e \emph{abstracts}. A BU fornece além do template geral .doc,
fornece um template para o resumo estendido, que também deve ser gerado usando
esse ambiente. Nos três tipos de resumo exigidos, o aluno é responsável por
incluir um parágrafo ao final contendo as Palavras-chave\footnote{Sugestão:
  \lacmd{vspace\{\lacmd{baselineskip}\}Palavras-chave: Kw1. Kw2. Kw3.}}: Regras
implementadas nesse ambiente:
\begin{itemize}
\item Espaçamento simples, fonte 12, sem indentação de parágrafos;
\item A formatação de \lacmd{section} torna-se igual à formatação de
  \lacmd{subsection}, para garantir cumprimento da formatação usada no template de
  resumo estendido;
\item Espaçamentos após títulos não são aplicados (resumo estendido);
\item A hifenização é alterada para \textit{brazil}, a menos que o autor tenha
  usado o ambiente abstract ou tenha usado Abstract, abstract ou ABSTRACT como
  argumento \texttt{<nome>} do ambiente.
\end{itemize}

\subsection{Comandos de outros pacotes modificados}

\paragraph*{\lacmd{listofalgorithms}} O nome é alterado para
\lacmd{ufscthesisalgorithmname} (Algoritmo em portugês e Algorithm em inglês) e
a formatação é ajustada para a BU. O uso do pacote algorithm2e é
auto-detectado. O pacote não será importado automáticamente e o suporte ao
pacote não é uma recomendação do seu uso. \textbf{Caso o comando não seja
  definido por nenhum outro pacote}, a presente classe o definirá utilizando o
pacote \texttt{newfloat}, em conjunto com um ambiente
\lacmd{begin\{algorithm\}...\lacmd{end\{algorithm\}}}.


\chapter{Mais exemplos de formatação}
\label{ch:ex}

Essa frase é verdadeira pois tem um \lacmd{cite} no final \cite{turing1937}. Essa
é mais verdadeira ainda pois tem um (\lacmd{cite\{turing1937,dijkstra1968\}}) no
final \cite{turing1937,dijkstra1968}. Já esta frase inofensiva usa
\lacmd{citeonline\{dijkstra1968\}} para citar \citeonline{dijkstra1968}
nominalmente. O trabalho de \citeonline{diffie1976} foi altamente influente
\cite{diffie1976}. Essa outra frase cita o trabalho que \citeonline{Saleem2018}
escreveu com outros 4 autores. Para algo completamente novo, veja um footnote
com url\footnote{\url{http://example.org/}}

Mais algumas citações de tipos específicos de documentos:
\begin{itemize}
\item \texttt{@inproceedings}: \citeonline{Ullman1989magic}. Jabuti
  \cite{Ullman1989magic}.
\item \texttt{@article}: \citeonline{Distefano2019}, framboesa \cite{Distefano2019};
\item \texttt{@book}: \citeonline{Abiteboul1995}, goiaba \cite{Abiteboul1995};
\item \texttt{@incollection}: \citeonline{Forgy1989}, melancia \cite{Forgy1989};
\item \texttt{@techreport}: \citeonline{rdf11}, figo \cite{rdf11}.
\end{itemize}

A lista abaixo mostra o efeito de \lacmd{autoref{}} com capítulos e (sub)seções.

\begin{itemize}
\item Há coisas na \autoref{ch:ex};
\item Há coisas na \autoref{sec:stuff};
\item Há coisas na \autoref{sec:more};
\item Há coisas na \autoref{sec:yet-more};
\item Há coisas na \autoref{sec:yet-another} (\abnTeX{} come um ``sub''
  intencionalmente no português e o mantem em inglês).
\end{itemize}

Citações são feitas com \laenv{citacao}. A BU faz as mesmas exigências que já
são o \textit{default} na classe \abnTeX\footnote{O alinhamento e o filete de
  notas de rodapé também não necessitou de modificações, além do tamanho da
  fonte. Essa frase não serve a nenhum propósito além de causar uma quebra de
  linha para que o alinhamento seja avaliado.}.

\begin{citacao}
  A elaboração do trabalho de conclusão de curso em nível de mestrado
  e de doutorado na UFSC deverá atender aos critérios e procedimentos
  estabelecidos nesta resolução normativa e em diretrizes
  estabelecidas pela Pró-Reitoria de Pós-Graduação e pelos Programas
  de Pós-Graduação.
\end{citacao}

Atenção! O template da BU deixa figuras e tabelas alinhadas à esquerda. No
entanto, o tutorial de Word disponibilizado pela BU diz que o \lacmd{caption\{\}}
(legenda acima) e o \lacmd{fonte\{\}} (o ``Fonte: o autor.'', abaixo) devem
respeitar o ``alinhamento da ilustração''. Esse tutorial apresenta apenas uma
ilustração alinhada à esquerda). O tutorial explicando a ABNT mostra uma figura
centralizada com legendas alinhadas a esquerda e com recuo até o começo da
figura. O autor do \texttt{.cls} se exime de qualquer culpa. Alinhe aqui (com
\lacmd{centering}, \lacmd{flushright} ou \lacmd{flushleft}) como mandar o seu
coração. Veja na \autoref{fig:logo} o efeito de se usar \lacmd{centering}.

\begin{figure}[t]
  \centering
  \caption{Logotipo da Universidade Federal de Santa Catarina.}
  \label{fig:logo}

  \includegraphics[width=.2\linewidth]{../logo-ufsc.pdf}
  \fonte{o autor.}
\end{figure}

\section{Coisas}
\label{sec:stuff}
Imagine alguma afirmação de alto valor científico aqui.

\subsection{Outras coisas}
\label{sec:more}
Olá! Eu vim do passado para te avisar que o texto de uma dissertação deve ser
impessoal e você não deveria tentar conversar com o leitor.

\subsubsection{Outras coisas mais}
\label{sec:yet-more}
Estudos demonstram que essa afirmação é falsa.

\subsubsubsection{Ainda outras coisas mais}
\label{sec:yet-another}
Fazer a grama verde, como? Novamente o jogo foi perdido. Opcionalmente, tudo
pode ser opcional. Recursos foram gastos com isso. Descubra a verdade nas
capitalizadas.
% Fiquei 15 minutos mais próximo da morte ao escrever isso. Você pode chegar
% ainda mais perto se tentar entender.

%%%%%%%%%%%%%%%%%%%%%%%%%%%%%%%%%%%%%%%%%%%%%%%%%%%%%%%%%%%%%%%%%%%%
%%% Elementos pós-textuais                                       %%%
%%%%%%%%%%%%%%%%%%%%%%%%%%%%%%%%%%%%%%%%%%%%%%%%%%%%%%%%%%%%%%%%%%%%

\postextual
\bibliography{userguide}

\end{document}

% LocalWords:  pdf ufscthesisrn abntex PPGCC Zambonin pull request GitHub PRs
% LocalWords:  alexishuf Issues alpha dropin template cls shellescape ZIP Git
% LocalWords:  UNIXlike Makefile git submodule shame on you Overleaf english cd
% LocalWords:  overview PósGraduação Prof BU pdfpages bug BibTeX URLs at bst or
% LocalWords:  Available citealf url xurl brazil embeddedlogo autocontido pt
% LocalWords:  lmodern arial opensource Latin Modern nocapautoref ufscthesis
% LocalWords:  working directory pdflatex tex dijkstra diffie Saleem Ullman rdf
% LocalWords:  magic Distefano Abiteboul Forgy default Word captions userguide
% LocalWords:  noplainurl noabntexcite begin dedicatoria end doc abstracts EQM
% LocalWords:  corretude indentação oneside memoir feature SAD EQD bookmarks of
% LocalWords:  nopretextualbookmark listoffigures table contents logodir OUTDIR
% LocalWords:  jobname jobname-logo latexmkoutdir outdir latexmk filecontents
% LocalWords:  write thesis such main
